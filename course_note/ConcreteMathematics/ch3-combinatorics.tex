\documentclass{article}
\usepackage{xeCJK}
\setCJKmainfont{Noto Serif CJK SC}

\usepackage{amsmath}
\usepackage{amssymb}

\usepackage[margin=0.2cm]{geometry}
\usepackage{ulem}
\usepackage{xcolor}
\usepackage{minted}
\usepackage{hyperref}

\newcommand{\N}{\mathbb{N}}
\newcommand{\Z}{\mathbb{Z}}
\newcommand{\Q}{\mathbb{Q}}
\newcommand{\R}{\mathbb{R}}

\newcommand{\blk}[1]{\left({#1}\right)}

\begin{document}

\section{binomial coefficient}

\subsection{binomial identitiese}

\[
	\begin{aligned}
		\forall \alpha\in \mathbb{R}\quad
		{(1+x)}^{\alpha}
		                         & =\sum_{n=0}^\infty \binom{\alpha}{n}x^n=\sum_{n=0}^\infty \dfrac{\alpha^{\underline n}}{n!} x^n \\
		\forall n\in \N^+, k\in \N\quad \binom{n}{k}
		                         & = \binom{n-1}{k-1}+\binom{n-1}{k}                                                               \\
		\forall n,k,i\in \N\ k\leq i\leq n\quad
		\binom{n}{k}\binom{k}{i} & = \binom{n}{i}\binom{n-i}{k-i}                                                                  \\
		\binom{-\alpha}{k}{(-1)}^k
		                         & =\dfrac{\prod_{i=0}^{k-1}(-\alpha-i)}{k!}{(-1)}^k
		=\dfrac{\prod_{i=0}^{k-1}(\alpha+i)}{k!}
		=\dfrac{{\alpha+k-1}^{\underline k}}{k!}
		=\binom{\alpha+k-1}{k}
	\end{aligned}
\]

\subsection{exercies involving binomial coefficient}
\textbf{TODO} 
\newpage

\section{finite calculus}

or \textbf{calculus of finite difference}

\subsection{formulas}

\begin{itemize}
	\item
	      \textbf{discrete derivative}: $g(x)=\Delta_x f(x)\iff g(x)=f(x+1)-f(x)$.\\
	      $g(x)$ is called the \textit{discrete derivative} or \textit{difference} of $f(x)$.\\
	      $\Delta$ is called the \textit{difference operator}.
	\item
	      Let $x^{\underline k}=\prod_{i=0}^{k-1}(x-i),\, x^{\overline k}=\prod_{i=0}^{k-1}(x+i)$
	      then $\Delta_x x^{\underline k}=k x^{\underline {k-1}}$\\
	      Quick notes: $\Delta_x c^x = (c-1)c^x,\ \Delta_x c = 0\ \Delta_x \binom{x}{k}=\binom{x}{k-1}$
	\item
	      $\Delta_x\blk{af(x)+bg(x)}=a\Delta_x f(x)+b\Delta_x g(x)$\\
	      The difference operator is a linear operator.
	\item
	      \textbf{discrete anti-derivative}: $\Delta_x f(x)=g(x)\iff \sum g(x)\delta x=f(x)+C$\\
	      $f(x)$ is called the \textit{discrete anti-derivative} or \textit{anti-difference} of $g(x)$\\
	\item
	      \textbf{discrete definite integral}: $\Delta_x f(x)=g(x) \iff \sum_{a}^b g(x)\delta x = \sum_{x=a}^{b-1} g(x)=f(b)-f(a)$
	\item
	      \textbf{finite integration by parts}: Abel partial summation formula\\
	      \[
		      \begin{aligned}
						A_i=\sum_{j=1}^{i}a_j \implies \sum_{i=1}^{n} a_i b_i
			       & = A_n b_n-\sum_{i=1}^{n-1} A_i (b_{i+1}-b_i)   \\
			      \sum u(x)\Delta v(x) \delta x
			       & =u(x)v(x)-\sum v(x+1) \Delta u(x) \delta x \\
			      u(x)\Delta v(x)
			       & = \Delta \blk{u(x)v(x)} - v(x+1)\Delta u(x)
		      \end{aligned}
	      \]
\end{itemize}

\subsection{exercies}

\textbf{TODO}


\end{document}
