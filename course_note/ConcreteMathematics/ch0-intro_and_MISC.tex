\documentclass{article}
\usepackage{xeCJK}
\setCJKmainfont{Noto Serif CJK SC}

\usepackage{amsmath}
\usepackage{amssymb}

\usepackage[margin=0.2cm]{geometry}
\usepackage{ulem}
\usepackage{xcolor}
% \usepackage{minted}
\usepackage{hyperref}

\begin{document}

\section{preface}
\begin{itemize}
	\item TOC
		\begin{itemize}
			\item 张江理工SI120 Discrete Mathematics, 2021Spring 例题,作业题,考试题.
			\item Concrete Mathematics 具体数学 例题,习题.
			\item 各种算法竞赛中相关的题目.
			\item 自己随意搜索到的相关知识,技巧,题目.
			\item 疑惑,思考,洞见.
		\end{itemize}
	\item guidelines
		\begin{itemize}
			\item 不求详尽,不求完整. 因为这仅仅是为自己做的笔记.
			\item 给出足够的参考资料信息,以便日后查阅.
			\item 排版没有明确规范,但不可随意乱搞.
		\end{itemize}
	\item reference
		\item resources from the Internet: wikipedia, Math StackExchange, MathOverflow \ldots
		\item textbook for SI120 course: \textit{Discrete Mathematics and Its Application}
		\item extra textbook: \textit{Concrete Mathematics}
\end{itemize}
\newpage

\tableofcontents

\newpage

\section{MISC}

这里会放一些 未归类/不易归类的东西 以及不在上文中划定的内容范围内的东西.


\end{document}
