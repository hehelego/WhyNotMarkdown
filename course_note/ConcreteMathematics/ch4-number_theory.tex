\documentclass{article}
\usepackage{xeCJK}
\setCJKmainfont{Noto Serif CJK SC}

\usepackage{amsmath}
\usepackage{amssymb}

\usepackage[margin=0.2cm]{geometry}
\usepackage{ulem}
\usepackage{xcolor}
\usepackage{minted}
\usepackage{hyperref}

\newcommand{\NN}{\mathbb{N}}
\newcommand{\ZZ}{\mathbb{Z}}
\newcommand{\QQ}{\mathbb{Q}}
\newcommand{\RR}{\mathbb{R}}
\newcommand{\CC}{\mathbb{C}}
\newcommand{\floor}[1]{\left\lfloor{#1}\right\rfloor}
\newcommand{\ceil}[1]{\left\lceil{#1}\right\rceil}

\newcommand{\ord}{\mathrm{ord}}
\newcommand{\idx}{\mathrm{idx}}

\begin{document}

\tableofcontents

\newpage

\section{初等数论内容概述}

引入一些群论的想法有助于学习初等数论.

\begin{itemize}
	\item 带余除法,整除,同余,整系数线性不定方程 几种基础表达形式的互相转化.
	\item 素数,互质,最大公约数, Beazout's theorem, Eculidean's algorithm.
	\item 算数基本定理.
	\item 数论函数, 数论函数求和问题.
	\item 线性同余方程组,CRT
	\item 模意义下的指数函数, 欧拉定理, 原根, 离散对数问题.
	\item 模意义下的幂函数, 二次剩余.
	\item MISC
	      \begin{itemize}
		      \item 费马大定理(Fermat's last theorem)
		      \item Wilson's theorem, $p\text{ is a prime} \iff (p-1)!\equiv 1\pmod p$
		      \item 素性判定和整数分解算法.
		      \item 素数分布, 素数定理.
		      \item 初等数论的密码学应用.
		      \item 狄利克雷生成函数 Dirichlet generating function.
		      \item 连分数.
	      \end{itemize}
\end{itemize}

\newpage

\section{Euler's totient function and Euler's theorem}

Let $\ZZ_n=\{[k]\mid k\in \ZZ \}$ where $[k]=\{ x\mid x\equiv k\pmod n \}$.
Consider $\ZZ_n^\ast = \{[k]\in \ZZ_n \mid k\perp n\}$, define $\varphi(n)=\left\vert|Z_n^\ast|\right\vert$.\\
The Euler's theorem claims that $(a,n)=1\implies a^{\varphi(n)}\equiv 1\pmod n$

\subsection{Evaluating Euler's totient function}

Let $U=\{p_1,p_2\ldots p_m\}$ be the set of prime factors of $n$\\
For every $S\in \rho(U)$, define
\begin{itemize}
	\item $f(S)=\left\vert
		      \left\{
		      k\in \ZZ_n
		      \mid
		      (p\in S \leftrightarrow p\mid k)
		      \right\}
		      \right\vert$
	\item $g(S)=\left\vert
		      \left\{
		      k\in \ZZ_n
		      \mid
		      (p\in S \rightarrow p\mid k)
		      \right\}
		      \right\vert=n\prod_{p\in S} \frac{1}{p}$
\end{itemize}

We have $g(S)=\sum_{S\subseteq T\subseteq U} f(T)$.
Applying the Mobius's inversion to get $f(S)=\sum_{S\subseteq T\subseteq U}{(-1)}^{|T|-|S|}g(T)$.\\

$$
	\begin{aligned}
		\varphi(n)
		 & =f(\varnothing)                                                                                        \\
		 & =n\sum_{S\subseteq \{p_1,p_2\ldots p_m\}}{(-1)}^{|S|}\prod_{p \in S}\frac{1}{p}                        \\
		 & =n\sum_{S\subseteq \{p_1,p_2\ldots p_m\}}\left(\prod_{p \in S}\frac{-1}{p} \prod_{j\not\in S} 1\right) \\
		 & =n\prod_{p\in S}(1-\frac{1}{p})
	\end{aligned}
$$


\subsection{The Euler's theorem}

$(\ZZ_n^\ast,\times)$ is a commutative group.
Consider $a\ZZ_n^\ast=\{[ak]\mid [k]\in \ZZ_n^\ast\}$
We claim that $a\ZZ_n^\ast = \ZZ_n^\ast$, otherwise\\

\[
	\begin{aligned}
		 & \exists x,y\quad x,y\perp n\land x\not\equiv y\pmod n                            \\
		 & ax\equiv ay\pmod n                                                               \\
		 & (a,n)=1\implies \exists a^{-1}\text{ s.t. } a^{-1}a\equiv aa^{-1}\equiv 1\pmod n \\
		 & a^{-1}ax\equiv x\equiv a^{-1}ay\equiv y\pmod n
	\end{aligned}
\]

Therefore

\[
	\begin{aligned}
		\prod_{k\perp n} k                & \equiv \prod_{k\perp n} ak & \pmod n \\
		a^{\varphi(n)} \prod_{k\perp n} k & \equiv \prod_{k\perp n} k  & \pmod n \\
		a^{\varphi(n)}                    & \equiv 1                   & \pmod n
	\end{aligned}
\]

For the second step: $\ZZ_n^\ast$ is closed under multiplication,
so $\prod_{k\perp n} k$ is also in $\ZZ_n^\ast$, which means it is invertible.


\newpage
\section{primitive root (modulo a prime)}

\subsection{definition}

\begin{itemize}
	\item \textbf{multiplicative order}:
	      $\forall x\in \ZZ_p^\ast\ \ord_p(x)=\min\{k\in \ZZ^+\mid x^k\equiv 1\pmod p\}$\\
				key property: $\ord_p(x) \mid \varphi(p)=p-1$, otherwise:
				\[
					1
					\equiv x^{\varphi(p)}
					\equiv x^{p-1}
					\equiv x^{\floor{\frac{p-1}{\ord(x)}\ord(x)}+(p-1)\bmod \ord(x)}
					\equiv x^{(p-1)\bmod \ord(x)}
					\not\equiv 1
					\pmod p
				\]
	\item \textbf{index}:
		$\forall x\in \ZZ_p^\ast,g\in \ZZ_p^\ast\ \idx_g(x)=\min\{k\in \NN\mid g^k\equiv x\pmod p\}$
	\item \textbf{primitive root}:
	      The generators of the multiplicative group $\ZZ_p^\ast$ are called the primitive roots modulo $p$.\\
	      For a prime number $p$, we say that $g$ is the primitive root modulo $p$ if $\ZZ_p^\ast = \{[1],[2]\ldots [p-1]\}=\{[g^0],[g^1]\ldots [g^{\varphi(p)-1}]=[g^{p-2}]\}$.\\
				$g$ is a primitive root modulo $p$ iff $\ord_p(g)=\varphi(p)=p-1$
\end{itemize}

\subsection{existance of primitive root}

\noindent \textbf{theorem: }For every odd prime number $p$, the primitive root modulo $p$ exists.\\
Therefore, for any odd prime number number $p$, the multiplicative group $\ZZ_p^\ast$ is a cyclic group.

\subsection{properties of primitive root}

\subsubsection{count of distinct primitive roots modulo a prime}

Let $g$ be a primitive root modulo $p$.
Then $\forall x\in \ZZ_p^\ast\ \exists!k\in\{0,1,2\ldots p-2\}\ g^p\equiv x\pmod p$\\
For $o=\ord_p(x)=\ord_p(g^k)$, we have
$x^o\equiv g^{ko}\equiv g^{0}\equiv 1\pmod p$\\
Let $ko=q(p-1)+r$ where $q\in \ZZ$ and $0\leq r<p-1$, 
so $g^{ko}\equiv g^{q(p-1)}\cdot g^{r}\equiv g^r\equiv 1\pmod p$.\\
Since the multiplicative order of $g$ is $p-1$, we must have $r=0$,
which implies $(p-1)\mid ko\iff ko\equiv 0\pmod{p-1}$.\\

When $k\perp (p-1)$, the multiplicative order of $x$ will be $p-1$, which means $x\equiv g^k$ is a primitive root modulo $p$.
Moreover, only when $(k,p-1)=1$, the smallest positive integer solution $o$ of i.e. $\ord_p(x)$ $ko\equiv 0\pmod {p-1}$ is $p-1$\\
Therefore, there are $\sum_{i=1}^{p-1}[\gcd(i,p-1)=1]=\varphi(p-1)$ distinct primitive roots modulo $p$.

\subsubsection{roots of unity}

solve the equation of $x$, $x^m \equiv 1\pmod p$.
Suppose that $x=g^k\ (0\leq k<p-1)$ is the solution, then $g^{km}\equiv 1\pmod p$.\\
Let $q=\floor{\frac{km}{p-1}}$ and $r=km\bmod {p-1}$, 
$g^{km}\equiv g^{(p-1)k}\cdot g^r\equiv g^r\equiv 1\pmod p$.\\
Therefore $r=0$ and $(p-1)\mid km\iff \frac{p-1}{\gcd(m,p-1)}\mid k$.

So, there are $(p-1,m)$ distinct $m$-th roots of unity in $\ZZ_p$.

\subsubsection{quadratic residule}


\noindent \textbf{definition}: $a$ is a quadratic residule modulo $p$ iff $\exists x\ x^2\equiv a\pmod p$.\\

Let $k=\idx_g(a)$, where $g$ is a primitive root.
\begin{itemize}
	\item $2\mid k$. $x\equiv g^{k/2}\pmod p$ is the square root of $a$.\\
		$a$ is a quadratic residule.
	\item $2\nmid k$. Suppose that $x\equiv g^r\pmod p$ is the square of $a$ modulo $p$\\
		$a\equiv x^2\equiv g^{2r}\equiv g^{k}\pmod p\iff g^{2r-k}\equiv 1\pmod p$.\\
		Let $q=\floor{\frac{2r-k}{p-1}},r=(2r-k)\bmod {(p-1)}$, we can show that $r=0$ and $(p-1)\mid (2r-k)$.\\
		However, $p-1$ is a even number while $2r-k$ is a odd number, this is impossible.\\
		Thus, $a$ is not a quadratic nonresidue.
\end{itemize}

Similarly we can show that $2\mid k$ if $a$ is a quadratic residule.
In conclusion, $a\text{ is a quadratic residule} \iff 2\mid \mathrm{idx}_g(a)$\\

This tells us that among $\ZZ_p^\ast=\{[g^0],[g^1]\ldots [g^{p-2}]\}$ there are $\frac{p-1}{2}$ quadratic residule. 
The property will be used when developing Cipolla's algorithm\footnote{An efficient algorithm for solving modular quadratic equation}.



\noindent \textbf{Euler's criterion}: $a^{\frac{p-1}{2}}\bmod p=\begin{cases}1 & a \text{ is a quadratic residule}\\ -1& a\text{ is a quadratic nonresidue}\\ 0 & p\mid a\end{cases}$

Let $k=\idx_g(a)$, where $g$ is a primitive root.
\begin{itemize}
	\item $a\text{ is a quadratic residule}\implies 2\mid k$\\
		Let $i=\frac{k}{2}$, $a^{\frac{p-1}{2}}\equiv g^{(p-1)i}\equiv 1\pmod p$
	\item $a\text{ is a quadratic nonresidule}\implies 2\nmid k$\\
		Let $k=2i+1$, $a^{\frac{p-1}{2}}\equiv g^{(p-1)i}\cdot g^{\frac{p-1}{2}}\equiv g^{\frac{p-1}{2}}\pmod p$\\
		Let $x\equiv g^{\frac{p-1}{2}}$, then $x^2\equiv 1\pmod p\iff (x+1)(x-1)\pmod p$\\
		$p$ has no factor besides itself, so $\left(p\mid (x+1)\right)\lor \left(p\mid (x-1)\right)$\\
		Since $g$ is a primitive root, $\ord_p(g)=p-1$ so $x\equiv g^{\frac{p-1}{2}}\equiv 1\pmod p$ can not be the case.\\
		Hence, $x\equiv g^{\frac{p-1}{2}}\equiv -1$
\end{itemize}

The other direction.
\begin{itemize}
	\item Suppose that $a^{\frac{p-1}{2}}\equiv g^{k\frac{p-1}{2}}\equiv 1\pmod p$\\
		Then, $p-1\mid k\frac{p-1}{2}\implies \left(2{(p-1)}^2\right)\mid k\implies 2\mid k$\\
		So $a$ is a quadratic residule.
	\item Suppose that $a^{\frac{p-1}{2}}\equiv g^{k\frac{p-1}{2}}\equiv -1\equiv g^{\frac{p-1}{2}}\pmod p$\\
		Then $g^{k\frac{p-1}{2}}{\left(g^{-1}\right)}^{\frac{p-1}{2}}\equiv g^{(k-1)\frac{p-1}{2}}\equiv 1$,\\
		which implies $2\mid (k-1)\implies 2\nmid k$\\
		So $a$ is a quadratic nonresidule.
\end{itemize}

In conclusion.
\begin{itemize}
	\item $a^{\frac{p-1}{2}}\equiv 1\pmod p$ iff $a$ is a quadratic residule.
	\item $a^{\frac{p-1}{2}}\equiv -1\pmod p$ iff $a$ is a quadratic nonresidule.
\end{itemize}

\end{document}
