\documentclass{article}
\usepackage{xeCJK}
\setCJKmainfont{Noto Serif CJK SC}

\usepackage{amsmath}
\usepackage{amssymb}

\usepackage[margin=0.2cm]{geometry}
\usepackage{ulem}
\usepackage{xcolor}
\usepackage{hyperref}

\usepackage{minted}

\begin{document}

\tableofcontents

\newpage
\section{Hall's marriage theorem}
Let $G=(V,E)=(X+Y,E)$ be a bipartite graph, $G$ has a complete matching from $X$ to $Y$ iff $\forall S\subseteq X\ |S|\leq |N(S)|$,\\
where $N_G(S)=\{y\in Y\mid \exists x\in S\ (x,y)\in E\}$.

\subsubsection{references \& links}

\begin{itemize}
	\item \href{https://en.wikipedia.org/wiki/Hall\%27s\_marriage\_theorem}{wikipedia: Hall's marriage theorem}
	\item \href{https://en.wikipedia.org/wiki/Deficiency_(graph_theory)}{wikipedia: deficiency}
\end{itemize}


\subsubsection{sufficiency}
($\Rightarrow$)\quad
Let $M=\{(x,f(x))\in E\mid x\in X\}\subseteq E$ be a complete matching.\\
$N_G(S)=\{y\in Y\mid \exists x\in S\ (x,y)\in E\}\supseteq \{f(x)\mid x\in S\}$.

\subsubsection{necessity}
($\Leftarrow$)\quad
Apply induction on $|X|$.
\begin{enumerate}
	\item When $|X|=1$, the necessity is hold.
	\item Suppose that the necessity is hold, for $|X|\leq k$
	\item For $|X|=k+1$, $\forall S\subseteq X\ |S|\leq |N_G(X)|$
	      \begin{itemize}
		      \item $\forall \varnothing\subsetneq S\subsetneq X\ |N_G(S)|>|S|$\\
		            Select any edge $e=(x,y)$. Remove the vertices $x,y$ and every edges $(u,v)$ to get $G'$ s.t. $x\in\{u,v\}\lor y\in\{u,v\}$.\\
		            In $G'$, $\forall \varnothing\subsetneq S\subseteq (X\setminus \{x\})\quad |N_{G'}(S)|\geq |N_{G}(S)|-1\geq |S|$.\\
		            So we can find a complete matching  $M':X'\to Y'$ in $G'$.\\
		            Thus A complete matching $M:X\to Y$ can generated by $M'\cup \{(x,y)\}$
		      \item $\exists \varnothing\subsetneq S\subsetneq X\ |N_G(S)|=|S|$\\
		            Let $G_0=(V_0,E_0),G_1=(V_1,E_1)$ be two induced subgraph of $G$, where $V_0=S+N_G(S),V_1=(X\setminus S)+ (Y\setminus N_G(S))$.
		            \begin{itemize}
			            \item For $G_0$: $\forall P\subseteq S\ N_{G_0}(P)=N_{G}(P)$ so $\forall P\subseteq S\ |N_{G_0}(P)|=|N_{G}(P)|\geq |P|$.
			            \item For $G_1$: We claim that $\forall Q\subseteq (X\setminus S)\ |N_{G_1}(Q)|\geq |Q|$.\\
			                  Otherwise, let $Q_0$ be a subset s.t. $|N_{G_1}(Q_0)|<|Q_0|$.\\
			                  Then $N_{G}(S\cup Q_0)=N_{G}(S)\cup N_{G}(Q_0)$, where $N_{G}(S)=N_{G_0}(S),\ N_{G}(Q_0)\subseteq (N_{G}(S)\cup N_{G_1}(Q)),\ N_{G_1(Q_0)}\cap N_G(S)=\varnothing$.\\
			                  which leads to $|N_{G}(S\cup Q_0)|=|\leq |N_G(S)|+|N_{G_0}(Q_0)| < |S|+|Q_0|$, however $|N_{G}(S\cup Q_0)|\geq |S\cup Q_0|=|S|+|Q_0|$.\\
			                  Therefore in $G_1$, $\forall Q\subseteq (X\setminus S)\ |N_{G_1}(Q)|\geq |Q|$.\\
		            \end{itemize}
		            We can find a complete matching by merging the complete matching in $G_0$ and $G_1$.
	      \end{itemize}
\end{enumerate}

\subsubsection{generalization}

In a bipartite graph $G=(X+Y,E)$
Let $\mathrm{def}_G(S)=|S|-|N_G(S)|$,
then the size of maximum matching in $G$ is $|X|-\max_{\varnothing\subseteq S\subseteq X}\mathrm{def}_G(S)$

\newpage
\section{Havel-Hakimi algorithm}

\subsubsection{references \& links}

\begin{itemize}
	\item \href{https://en.wikipedia.org/wiki/Havel\%E2\%80\%93Hakimi\_algorithm}{wikipedia: Havel-Hakimi algorithm}
	\item \href{https://en.wikipedia.org/wiki/Erd\%C5\%91s\%E2\%80\%93Gallai\_theorem}{}
\end{itemize}

\subsubsection{the theorem}

Given a list of non-negative integers $(d_1,d_2\ldots d_n)$ where $d_1\geq d_2\geq d_3\ldots d_n\geq 0$.\\
Is there a undirected simple\footnote{no loops $(v,v)$, no dup-edges $k\times (v,v)$} graph $G=(V,E)$ s.t. the degree sequence of $G$ is $(d_1,d_2\ldots d_n)$?

\begin{minted}{python}
from typing import List
def check(deg_seq: List[int]) -> bool:
    if len(deg_seq)==0:
        return True

    head,seq = deg_seq[0],deg_seq[1:]
    if len(seq)>=head:
        for i in range(head):
            seq[i]=seq[i]-1
        seq = sorted(seq, reverse=True)
        return seq[-1]>=0 and check(seq)
    return False
\end{minted}

\subsubsection{proof}

\textbf{TODO}



\newpage
\section{graph isomorphism invariant}

\subsection{degree sequence}

$G\cong H\implies f(G)=H(G)$, where $f(G)=\mathrm{multi-set}\{\deg_G(v)\mid v\in G\}$.\\
This is a necessary but not sufficient condition, see the following example, where $H,G$ have the same degree sequence but are not isomorphic to each other.

\begin{minted}{C}
//!/usr/bin/dot
// in graphviz dot
\graph G{
	1 -- 2
	2 -- 3
	3 -- 4
	4 -- 5
	2 -- x
}
\graph H{
	1 -- 2
	2 -- 3
	3 -- 4
	4 -- 5
	3 -- x
}
\end{minted}






\end{document}
