\documentclass{article}

\usepackage{amsmath, amsthm, amssymb}

\title{Quick Note on Omega-Automaton}
\author{hehelego}
\date{Dec 25, 2023}

\theoremstyle{definition}
\newtheorem{theorem}{Theorem}
\newtheorem{lemmma}{Lemma}
\newtheorem{definition}{Definition}
\newtheorem{subdef}{Definition}[definition]

\setlength\parindent{0pt}

\begin{document}
\maketitle

\newcommand{\exinf}{\stackrel{\infty}{\exists}}
\newcommand{\NN}{\mathbb{N}}
\newcommand{\TT}{\mathbf{T}}
\newcommand{\FF}{\mathbf{F}}
\newcommand{\BOOL}{\{\TT,\FF\}}

Define the \(\exinf\) quantifier, which reads as ``exists infinitely many''.
\[
	\exinf i\in\NN. p(i) = \forall n\in\NN . \exists i > n . p(i)
\]
All occurrences of \(\Sigma\) in this note refers to a finite alphabet set.

\(\epsilon\) denotes a \emph{null letter}, an automaton is said to consume a \(\epsilon\) letter when when it is not consuming any input.

\section{\(\omega\)-Words and \(\omega\)-Automata}

\begin{definition}
	A \(\omega\)-word over a \(\Sigma\) is a mapping from \(\NN\) to \(\Sigma\). In other words, a \(\omega\)-word is an infinite sequence \(w=w_0w_1\ldots\) where \(w_i\in\Sigma\) for every \(i\in\NN\).
\end{definition}
\begin{subdef}
	A \(\omega\)-language \(L\) is a subset of all \(\omega\)-words, \(L\subseteq\Sigma^\omega\) where \(\Sigma^\omega=\{w_0w_1\ldots \mid \forall i\in\NN .\ w_i\in\Sigma\}\) is the set of all \(\omega\)-words.
\end{subdef}
\begin{subdef}
	For every \(w\in\Sigma^\omega\) define \(\inf(w) = \{x\in\Sigma | \exinf i\in\NN . w_i=x \}\) to be the letters that occurred infinitely many times in \(w\).
\end{subdef}

\begin{definition}
	A \(\omega\)-automaton is a tuple \(A=(Q,\Sigma,\delta,Q_0,\mathsf{Accept})\), where
	\begin{itemize}
		\item \(Q\) is a \emph{finite} state set.
		\item \(\Sigma\) is a finite alphabet set.
		\item \(\delta \subseteq Q\times\Sigma\times Q\) or \(\delta: Q\times \Sigma\to \mathcal{P}(Q)\) is the state transition relation/function.
		\item \(\emptyset \subsetneq Q_0\subseteq Q\) is the set of initial states.
		\item \(\mathsf{Accept}:Q^\omega\to\BOOL\) is an specific accepting condition predicate.
	\end{itemize}
	An \(\omega\)-automaton is said to be \emph{deterministic} if \(|Q_0|=1\) and
	\[
		\forall (q,a)\in Q\times \Sigma .
		\left|\left\{q'\in Q\mid (q,a,q')\in \delta\right\}\right|\leq 1
	\]
\end{definition}

\begin{subdef}
	A run of \(w\in\Sigma^\omega\) on \(A=(Q,\Sigma,\delta,Q_0,\ast)\) is \(\rho\in Q^\omega\), where \(\rho_0\in Q_0\) and \((\rho_i,w_i,\rho_{i+1})\in\delta\) for every \(i\in\mathbb{N}\).
	The predicate \(\operatorname{run}(\rho,w,A)\) is defined to be true if \(\rho\) is a valid run of \(w\) on \(A\).
\end{subdef}

\begin{subdef}
	The language recognized by \(A=(Q,\Sigma,\delta,Q_0,\mathsf{Accept})\) is
	\[
		\mathcal{L}(A) = \{w\in\Sigma^\omega\mid
		\exists \rho\in Q^\omega .
		\operatorname{run}(\rho,w,A) \land \mathsf{Accept}(\rho)\}
	\]
	A \(\omega\)-automaton \(A\) recognizes a \(\omega\)-word \(w\) if \(w\) has an accepting run \(\rho\) on \(A\).
\end{subdef}

\begin{subdef}
	An \(\omega\)-automaton with \(\epsilon\) transition is a tuple \(A'=(Q,\Sigma,\delta\cup\delta_{\epsilon},Q_0,\mathsf{Accept}\) where \(\delta\subseteq Q\times\Sigma\times Q\) and \(\delta'\subseteq Q\times\{\epsilon\}\times Q\).

\end{subdef}

\section{B\"uchi Automata}

A B\"uchi automaton is a \(\omega\)-automata \(A=(Q,\Sigma,\delta,Q_0,F)\), where \(F\subseteq Q\) and \(\mathsf{Accept}(\rho) = \inf(\rho)\cap F \neq \emptyset\).
So a B\"uchi automaton accepts a run \(\rho\) if at least one good state \(g\in F\) is infinitely often visited.

The power of NBA is strictly stronger than DBA.
The language of NBAs are \(\omega\) regular languages.

\section{Omega Regular Language}

\begin{definition}
\end{definition}


\section{Omega-Automata}


\section{Set Operations}

\section{Complement of a NBA}

\end{document}
