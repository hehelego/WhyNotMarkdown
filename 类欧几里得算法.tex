\documentclass{article}
\usepackage{hyperref}
\usepackage{amsmath}
\usepackage{amssymb}
\usepackage{xeCJK}
\setCJKmainfont{Noto Serif CJK SC}

\begin{document}

\newcommand{\FLR}[1]{{ \left\lfloor #1 \right\rfloor }}

\section{problem statement}

\noindent 给定正整数$a,b,c,n$,
计算$\displaystyle f(a,b,c,n)=\sum_{x=0}^n \FLR{\frac{ax+b}{c}}$\\
即第一象限内在$y=ax+b$下方的整点计数.

\section{``类欧几里得算法''}

\subsection*{case 1: $a\geq c \texttt{ or } b\geq c$}

\[
	\begin{aligned}
		f(a,b,c,n)
			&=\sum_{x=0}^n \FLR{\frac{ax+b}{c}}\\
			&=\sum_{x=0}^n
				\FLR{
					\FLR{\frac{a}{c}} x + \FLR{\frac{b}{c}}+
					\frac{(a\bmod c)c+(b\bmod c)}{c}
				}\\
			&=\FLR{\frac{a}{c}}\frac{n(n+1)}{2} + \FLR{\frac{b}{c}}(n+1) +
				\sum_{x=0}^n \FLR{\frac{(a\bmod c)c+(b\bmod c)}{c}}\\
			&=\FLR{\frac{a}{c}}\frac{n(n+1)}{2} + \FLR{\frac{b}{c}}(n+1)
				+ f(a\bmod c,b\bmod c,c,n)\\
	\end{aligned}
\]


\subsection*{cans 2: $a,b < c$}

\[
	\begin{aligned}
		f(a,b,c,n)
			&=\sum_{x=0}^n \FLR{\frac{ax+b}{c}}
			 =\sum_{x=0}^n \sum_{i=1}^\FLR{\frac{an+b}{c}}
				\left[ i \leq \FLR{\frac{ax+b}{c}} \right]
			 =\sum_{x=0}^n \sum_{i=1}^{\left\lfloor\frac{an+b}{c}\right\rfloor}
			 	\left[ i \leq \frac{ax+b}{c} \right]\\
			&=\sum_{i=1}^\FLR{\frac{an+b}{c}}
					\sum_{x=0}^n \left[ x\geq \frac{ic-b}{a} \right]
			=\sum_{i=1}^\FLR{\frac{an+b}{c}}
					(n+1)-\sum_{x=0}^n \left[ x < \frac{ic-b}{a}\right]\\
	(*) &=\sum_{i=1}^\FLR{\frac{an+b}{c}}
					(n+1)-\sum_{x=0}^n \left[ x \leq \frac{ic-b-1}{a}\right]\\
			&=\sum_{i=1}^\FLR{\frac{an+b}{c}}
					(n+1)-\sum_{x=0}^n \left[ x \leq \FLR{\frac{ic-b-1}{a}}\right]\\
			&=\FLR{\frac{an+b}{c}}(n+1)
				-\sum_{i=0}^{\FLR{\frac{an+b}{c}} -1}
				\sum_{x=0}^n \left[ x \leq \FLR{\frac{ic+c-b-1}{a}}\right]\\
			&=\FLR{\frac{an+b}{c}} n
				-\sum_{i=0}^{\FLR{\frac{an+b}{c}}- 1}
				\sum_{x=1}^n \left[ x \leq \FLR{\frac{ic+c-b-1}{a}}\right]\\
			&=\FLR{\frac{an+b}{c}}n-f\left(c,c-b-1,a,\FLR{\frac{an+b}{c}} -1\right)
	\end{aligned}
\]

其中$(\ast)$,的处理比较特殊,主要的困难在$a\mid ic-b$时如何避免算入$\frac{ic-b}{a}$.\\
设$r=ic-b\bmod a$,如果$r=0$,这个减1会帮助我们卡掉$x=\frac{ic-b}{a}$这个不该有贡献的值, 而$r>0$时,则没有影响.\\
也可以这样考虑: $ax < ic-b\iff ax \leq ic-b-1$.


\section{complexity}

设$a,b,c=O(n)$,分析上述算法的时间复杂度.

\subsection*{Lemma 0}

\[ a\geq b \implies (a\bmod b) \leq \frac{a}{2} \]

\[
\begin{array}{c|l}
	b\leq a\leq 2b & a\bmod b=a-b\quad a\leq 2b\implies \frac{a}{2}\leq b\implies \frac{a}{2}\geq a-b\\
	a > 2b & a\bmod b < b \leq \frac{a}{2}
\end{array}
\]


观察$f(a,b,c,n)$的$a,c$两个参数.\\
如果$a\geq c$,那么递归求解一个$a'=a\bmod c,c'=c$的和; 有$a' < c'$ .\\
如果$a<    c$,那么递归求解一个$a'=c,c'=a$的和; 有$a' \geq c'$ .\par
所以,每进行两次递归求解,参数$a$减半. 故$O(\log n)$时间化规到$a=0$的退化形式,这时可以$O(1)$求解.\par
于是,算法的时间复杂度是$O(\log n)$.

\section{generalized algorithm}

\noindent 给定$a,b,c,n,p,q$求解$\displaystyle \sum_{x=0}^n
{\FLR{\frac{ax+b}{c}}}^{p}
x^q$.\\
其中$p,q$是与$n$无关的小常数\footnote{比如LOJ上面有$p+q\leq 10,\ a,b,c,n\leq 10^9$}


\subsection*{case 1: $a\geq c\texttt{ or }b\geq c$}

\[
	\begin{aligned}
		S
		&=\sum_{x=0}^n \FLR{\FLR{\frac{a}{c}} x+\FLR{\frac{b}{c}}+\frac{(a\bmod c)x+(b\bmod c)}{c}}^p x^q\\
		&=\sum_{x=0}^n
			{\left(
				\FLR{\frac{a}{c}}x + \FLR{\frac{b}{c}} + \FLR{\frac{(a\bmod c)x+(b\bmod c)}{c}}
			\right)}^p x^q\\
		&=\sum_{x=0}^n
				\sum_{\begin{subarray}{c}0\leq k_1,k_2,k_3\leq p\\ k_1+k_2+k_3=p\end{subarray}}
					\frac{p!}{k_1!k_2!k_3!}
					\FLR{\frac{a}{c}}^{k_1}
					\FLR{\frac{b}{c}}^{k_2}
					\FLR{\frac{(a\bmod c)x+(b\bmod c)}{c}}^{k_3}
					\ x^{k_1 + q}\\
		&=\sum_{\begin{subarray}{c}0\leq k_1,k_2,k_3\leq p\\ k_1+k_2+k_3=p\end{subarray}}
				\frac{p!}{k_1!k_2!k_3!}
				\FLR{\frac{a}{c}}^{k_1}
				\FLR{\frac{b}{c}}^{k_2}
				\sum_{x=0}^n
					\FLR{\frac{(a\bmod c)x+(b\bmod c)}{c}}^{k_3}
					\ x^{k_1 + q}\\
	\end{aligned}
\]

\subsection*{case 2: $a,b < c$}

\[
	\begin{aligned}
		S
		&=\sum_{x=0}^n \FLR{\frac{ax+b}{c}}^p x^q\\
	\end{aligned}
\]

\newpage
\appendix

\section{code}

\section{reference}

\begin{itemize}
	\item \href{https://oi-wiki.org/math/euclidean/}{OI-wiki: 类欧几里得算法}
	\item \href{https://blog.csdn.net/qq_39972971/article/details/94618394}{XuYiXuan blog: LOJ138}
	\item \href{https://loj.ac/p/138}{LOJ138 类欧几里得算法}
\end{itemize}

\end{document}



